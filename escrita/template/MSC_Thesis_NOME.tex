\documentclass[
  oneside,
  11pt, a4paper,
  footinclude=true,
  headinclude=true,
  cleardoublepage=empty
]{scrbook}
\makeatletter
\DeclareOldFontCommand{\rm}{\normalfont\rmfamily}{\mathrm}
\DeclareOldFontCommand{\sf}{\normalfont\sffamily}{\mathsf}
\DeclareOldFontCommand{\tt}{\normalfont\ttfamily}{\mathtt}
\DeclareOldFontCommand{\bf}{\normalfont\bfseries}{\mathbf}
\DeclareOldFontCommand{\it}{\normalfont\itshape}{\mathit}
\DeclareOldFontCommand{\sl}{\normalfont\slshape}{\@nomath\sl}
\DeclareOldFontCommand{\sc}{\normalfont\scshape}{\@nomath\sc}
\makeatother
\usepackage[T1]{fontenc}
\usepackage{amsmath}
\usepackage{amssymb}
\usepackage{amsthm}
\usepackage{dissertation}
\usepackage[utf8]{inputenc}
\usepackage{url}
\usepackage{xspace}
\usepackage{apalike}
\usepackage{float}
\usepackage{import}
\usepackage{natbib}

\usepackage{multicol}
\usepackage{lscape}
\usepackage{enumerate}
\usepackage{parcolumns}
\usepackage{enumitem}
\usepackage{pdfpages}
\usepackage{tabularx}
\usepackage{array}
\newcolumntype{M}[1]{>{\centering\arraybackslash}m{#1}}
\newcolumntype{L}{>{\centering\arraybackslash}X}
\renewcommand\tabularxcolumn[1]{m{#1}}

\usepackage{color}
\definecolor{codegreen}{rgb}{0,0.6,0}
\definecolor{codegray}{rgb}{0.5,0.5,0.5}
\definecolor{codepurple}{rgb}{0.58,0,0.82}
\definecolor{backcolour}{rgb}{0.95,0.95,0.92}
\usepackage{listings}
\lstdefinestyle{mystyle}{
    backgroundcolor=\color{backcolour},
    commentstyle=\color{codegreen},
    keywordstyle=\color{magenta},
    numberstyle=\tiny\color{codegray},
    basicstyle=\footnotesize,
    breakatwhitespace=false,
    breaklines=true,
    captionpos=b,
    keepspaces=true,
    numbers=left,
    numbersep=5pt,
    showspaces=false,
    showstringspaces=false,
    showtabs=false,
    tabsize=2
}
\lstset{style=mystyle}

% ACRONYMS ----------------------------------------------------

\usepackage[acronym,nonumberlist,nomain]{glossaries}
\defglsdisplayfirst[\acronymtype]{\emph{#1#4}}
\glossarystyle{listgroup}
% set the name for the acronym entries page
\renewcommand{\glossaryname}{Acronyms}
\makeglossaries

% TITLE -------------------------------------------------------
\titleA{Dissertation title} %if title is to extensive, use titleB
%\titleB{Second part of title} % (if any)
%\subtitleA{}
%\subtitleB{Second part of Subtitle} % (if any)

% Author
\author{Full name}

% Supervisor(s)
\supervisor{Supervisor name}
\cosupervisor{Co-supervisor name} %comment in case of none

% University (uncomment if you need to change default values)
% \def\school{Escola de Engenharia}
% \def\department{Departamento de Inform\'{a}tica}
% \def\university{Universidade do Minho}
% \def\masterdegree{Informatics Engineering}

% Date
\date{\myear} % change to text if date is not today

% Keywords
%\keywords{master thesis}

% Glossaries & Acronyms
%\makeglossaries  %  either use this ...
%\makeindex        % ... or this

% Define Acronyms
%example
\newacronym{aam}{AAM}{Analytical Activity Method}

\glsaddall[types={\acronymtype}]

\ummetadata % add metadata to the document (author, publisher, ...)

\begin{document}
        % Cover page ---------------------------------------
        \umfrontcover
        \umtitlepage

        % Add acknowledgements ----------------------------
\import{sections/}{Statements.tex}
\import{sections/}{Agradecimentos.tex}
\import{sections/}{Abstract.tex}


        % Summary Lists ------------------------------------
        \tableofcontents
        \listoffigures
        \listoftables
        %\lstlistoflistings
        %\listofabbreviations
        \printglossary[type=\acronymtype]
        \clearpage
        \thispagestyle{empty}
        \pagenumbering{arabic}


%%%%%%%%%%%%%%%%%%%%%%%%%%%%%%%%%%%%%%%%%%%%
%%%%%%%%%%%%%%%%%%%%%%%%%%%%%%%%%%%%%%%%%%%%
%%%%%%%%%---- IMPORTS ----%%%%%%%%%%%%%%%%%%
%%%%%%%%%%%%%%%%%%%%%%%%%%%%%%%%%%%%%%%%%%%%
%%%%%%%%%%%%%%%%%%%%%%%%%%%%%%%%%%%%%%%%%%%%

\import{sections/}{Chapt_Intro.tex}
\import{sections/}{Chapt_stateofart.tex}
\import{sections/}{Chapt_proposal.tex}
\import{sections/}{Chapt_development.tex}
\import{sections/}{Chapt_conclusion.tex}

%%%%%%%%%%%%%%%%%%%%%%%%%%%%%%%%%%%%%%%%%%%%
%%%%%%%%%%%%%%%%%%%%%%%%%%%%%%%%%%%%%%%%%%%%
%%%%%%%%%%%%%%%%%%%%%%%%%%%%%%%%%%%%%%%%%%%%
%%%%%%%%%%%%%%%%%%%%%%%%%%%%%%%%%%%%%%%%%%%%



        \bookmarksetup{startatroot}
        \addtocontents{toc}{\bigskip}
        \bibliography{MSc_Thesis_NOME}

        \umappendix{Appendix}

        \chapter{Support material; Listings}
        Auxiliary results Details of results which are not main-stream (like Specifications, Listings, Application Screenshots,
        Forms, Surveys (Questionnaires), complementary tables or graphics)
        whose length would compromise readability of main text,
        shall be included in the Appendix part divided into 1 or more chapters.

        %Anyone using \Latex\ should consider having a look at \TUG,
        %the \tug{\TeX\ Users Group}


        % Back Cover -------------------------------------------
        %\umbackcover{
        %NB: place here information about funding, FCT project, etc in which the work is framed. Leave empty otherwise.
        %}
  

\end{document}